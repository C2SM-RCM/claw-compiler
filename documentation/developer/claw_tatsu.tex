\chapter{CLAW TATSU \xcodemlf AST abstraction and manipulation library}
\label{chapter:clawtatsu}
As the \xcodemlf \gls{ir} is based on the XML format, it can be manipulated with
any language that can read and write XML files. It can even be manipulated by
hand depending on the user's knowledge of the \xcodemlf \gls{ir} specification.
To ease this task, an AST abstraction and manipulation library is included 
in the \cx2x program. This library helps to traverse the \gls{ast}, access and
update the type table, add, modify or delete node.

This section is currently very simple but will be enhanced in the future version
of this manual.

\section{The Xnode class}
The main class of this library is the \lstinline|Xnode| class that is the base
of any node in the AST. 
Any specialization of the a node like \lstinline|XcodeML|, 
\lstinline|XcodeProgram|, \lstinline|Xmod| or \lstinline|XbasicType| inherits 
from \lstinline|Xnode|.
This class has basic methods to perform the following actions:
\begin{itemize}
\item Query, add, update or remove an attribute of a node 
      (See Listing \ref{lst:xnode_attr}).
\item Query, add or remove children of a node (See Listing \ref{lst:xnode_node}).
\item Clone a node (See Listing \ref{lst:xnode_clone}).
\item Match nodes in the siblings, descendant or ancestor of a node 
      (See Listing \ref{lst:xnode_match}).
\end{itemize}

Each \lstinline|Xnode| instance has a specific opcode that match the \xcodemlf 
element tag.
This opcode is also used when creating new node or matching nodes in a tree.
All the opcode are members of the \lstinline|Xcode| class.
The \lstinline|Xattr| class contains all the members to access and match 
attribute of a node. 

\begin{lstlisting}[label=lst:xnode_attr, language=Java, caption=XtypeTable]
// Gather a node
Xnode n = ...
// Check if a node has an attribute
boolean hasAttributeAllocatable = n.hasAttribute(Xattr.IS_ALLOCATABLE);
// Get value from a boolean attribute
boolean attributeAllocatable = n.getBooleanAttribute(Xattr.IS_ALLOCATABLE);
// Get value from an attribute
String type = n.getAttribute(Xattr.TYPE);
// Set value to a boolean attribute
n.setAttribute(Xattr.IS_ALLOCATABLE, true);
// Set value to an attribute
n.setAttribute(Xattr.TYPE, "VALUE_OF_TYPE");
// Remove an attribute
n.removeAttribute(Xattr.IS_ALLOCATABLE);
\end{lstlisting}

\begin{lstlisting}[label=lst:xnode_node, language=Java, caption=XtypeTable]
// Gather a node
Xnode n = ...
// Get first child
Xnode fChild = n.firstChild();
// Get last child
Xnode lChild = n.lastChild();
// Iterate through children
Xnode crt = n.firstChild();
while(crt != null) {
  crt = crt.nextSibling()
}
// Add a child
Xnode intConst = xcodeml.createIntConstant(10);
n.append(xcodeml.createNode(intConst);
\end{lstlisting}

\begin{lstlisting}[label=lst:xnode_clone, language=Java, caption=XtypeTable]
// Gather a node
Xnode n = ...
// Get first child
Xnode clone = n.cloneNode();
\end{lstlisting}

\begin{lstlisting}[label=lst:xnode_match, language=Java, caption=XtypeTable]
// Gather a node
Xnode n = ...
// Match all arrayRef nodes in sub-tree
List<Xnode> arrayRefs = n.matchAll(Xcode.FARRAYREF);
// Match all arrayRef nodes in the siblings of the node
List<Xnode> arrayRefs = n.matchSibling(Xcode.FARRAYREF);
// Match first arrayRef node in the direct descendant of the node
Xnode arrayRef = n.matchDirectDescendant(Xcode.FARRAYREF);
\end{lstlisting}

\section{Access the type table}
The type table of an \xcodemlf translation unit is a child node of the root. 
The \lstinline|XcodeML| class gives access to this type table. 
\lstinline|XcodeProgram| and \lstinline|Xmod| are child class of 
\lstinline|XcodeML|. 
In the example in Listing \ref{lst:xtype_table}, the code is checking
every function definitions to find declaration of allocatable arrays.
On line 9, the type table is accessed to check if the type of the 
declaration is a \lstinline|FbasicType|. If so, the type is retrieved from
the type table (line 12) and its information can be queried to check its
nature.

\begin{lstlisting}[label=lst:xtype_table, language=Java, caption=XtypeTable]
public boolean analyze(XcodeProgram xcodeml, Translator translator) {
  // Get all function definitions in the translation unit
  List<XfunctionDefinition> fctDefs = xcodeml.getAllFctDef();
  for(XfunctionDefinition fctDef : fctDefs) {
    // Get the declarations of the function
    List<Xnode> declarations = fctDef.getDeclarationTable().values();
    for(Xnode decl : declarations) {
      if(decl.is(Xcode.VAR_DECL)) {
        if(!xcodeml.getTypeTable().isBasicType(decl)) {
          continue; // Only FbasicType can be array
        }
        XbasicType bt = xcodeml.getTypeTable().getBasicType(decl);
        if(bt != null && bt.isArray() && bt.isAllocatable()) {
          System.out.println(decl.decl.matchSeq(Xcode.NAME).value());
        }
      }
    }
  }
}
\end{lstlisting}



\section{Traverse the \gls{ast}}
Listing \ref{lst:ast_next_sib} shows a simple way to go from a pragma and visit 
its next siblings.

\begin{lstlisting}[label=lst:ast_next_sib, language=Java, caption=XcodeML/F AST traverse]
public void transform(XcodeProgram xcodeml, Translator translator,
                      Transformation other)
{
    Xnode crt = _claw.getPragma();
    while(crt != null){
      System.out.println("Current node: " + crt.opcode());
      crt = crt.nextSibling();
    }
}
\end{lstlisting}

\section{Add nodes}
A new node can be easily added in the AST as shown in the Listing 
\ref{lst:add_node}. On line 4, the new "node" object is created by
passing the "opcode" of the node. This code refers the the XcodeML/F 
specification defined in the \lstinline!Xcode! enumeration.

On line 5, a new value is assigned to the node. \lstinline!FpragmaStatement! 
node accept text values.

Finally, on line 6, the node is inserted in the AST. In this case, 
the \lstinline!insertAfter! will add the new node after the current node.

\begin{lstlisting}[label=lst:add_node, language=Java, caption=XcodeML/F add node example]
public void transform(XcodeProgram xcodeml, Translator translator,
                      Transformation other)
{
    Xnode myNewNode = xcodeml.createNode(Xcode.FPRAGMASTATEMENT);
    myNewNode.setValue("omp parallel");
    _claw.getPragma().insertAfter(myNewNode);
}
\end{lstlisting}

\section{Modify nodes}

\begin{lstlisting}[label=lst:update_node, language=Java, caption=XcodeML/F update node example]
public void transform(XcodeProgram xcodeml, Translator translator,
                      Transformation other)
{
    Xnode pragma = _claw.getPragma();
    // Get current values from the node
    String oldValue = pragma.value();
    int line = pragma.lineNo();
    // Update the value directly in the AST
    pragma.setValue("acc routine seq");
    pragma.setLineNo(line + 1);
}
\end{lstlisting}

\section{Delete nodes}
\begin{lstlisting}[label=lst:delete_node, language=Java, caption=XcodeML/F delete node example]
public void transform(XcodeProgram xcodeml, Translator translator,
                      Transformation other)
{
    Xnode pragma = _claw.getPragma();
    XnodeUtil.safeDelete(pragma); // Delete the node in the AST
}
\end{lstlisting}

%TODO more complex example.
