\chapter{CLAW Primitives}
\label{chapter:primitives}
CLAW transformations, low-level or high-level, are built on various smaller 
blocks that can be re-used in other transformations. Those blocks are called
primitives as they define a well constrained test or transformation on the
\xcodemlf AST. These primitives block use the \xcodemlf AST abstraction library
detailed in Chapter \ref{chapter:astmanip}.
The different primitives available in the CLAW framework are listed in this
chapter.

 
%TODO complete the chapter with all primitives
\textit{!!! CURRENTLY NOT COMPLETE !!!}.

\section{Loop primitive}
In this section, primitive transformations applicable on do statements are 
detailed.

\subsection{Loop fusion (merge)}
Loop fusion or merge is the process of merging the body of a slave loop into
the one of a master loop.
In \xcodemlf, a \lstinline|FdoStatement| has a child \lstinline|body| node.
This \lstinline|body| node has 0 to N children. Performing a merge of two
\lstinline|FdoStatement| is done by shifting all the element of the slave
\lstinline|FdoStatement->body| to the master \lstinline|FdoStatement->body|.

\subsubsection{API}
Signatures of the available methods to perform a loop fusion are shown in
Listing \ref{lst:pri_merge}. The first is used to perform a simple fusion 
on two simple \lstinline|FdoStatement|. The second uses 
\lstinline|NestedDoStatement| objects and will perform the merge of the 
inner most bodies.

\begin{lstlisting}[label=lst:pri_merge, language=java, title=Loop.java - merge]
public static void merge(Xnode master, Xnode slave);
public static void merge(NestedDoStatement master, NestedDoStatement slave);
\end{lstlisting}


\subsection{Loop reorder}
Loop reorder is the process of swapping iteration range between nested loops.
The current implementation support up to 3 nested loops to be reordered.
\subsubsection{API}
Signature of the available method to perform a loop reorder is shown in 
Listing~\ref{lst:pri_reorder}. First parameter is a 
\lstinline|NestedDoStatement| object representing the nested loops. The second
parameter is the list of induction varibale in their new order.

\begin{lstlisting}[label=lst:pri_reorder, language=java,title=Loop.java - reorder]
public static void reorder(NestedDoStatement nestedGroup,
                           List<String> newInductionVarOrder)
\end{lstlisting}


\section{Field primitive}
%TODO


\subsection{Scalar and arrays promotion}
Promotion of a scalar or array field is defined as a primitive transformation.
\lstinline|Field.promote| is the only method to be called to perform
this transformation on a field.

A promotion is defined by various elements. First, the dimension definition 
describes a dimension that will be used for the promotion.
In a promotion process, 1 to N dimension are added to an existing field type. 
Each dimension definition can describe its insertion position regarding the 
existing dimensions.


\begin{lstlisting}[language=fortran]
REAL, DIMENSION(1:10,1:20) :: a

! Some dimensions used to illustrate the promotion process
! Dimension description ndim(1:30)
! Dimension description zdim(1:30)

! Promoted with ndim before existing dimension
REAL, DIMENSION(ndim,1:10,1:20) :: a
! Promoted with ndim in middle existing dimensions
REAL, DIMENSION(1:10,ndim,1:20) :: a
! Promoted with ndim after existing dimensions
REAL, DIMENSION(1:10,1:20,ndim) :: a

! Promoted with ndim and zdim before existing dimension
REAL, DIMENSION(ndim,zdim,1:10,1:20) :: a
! Promoted with ndim and zdim in middle existing dimensions
REAL, DIMENSION(1:10,ndim,zdim,1:20) :: a
! Promoted with ndim,zdim after existing dimensions
REAL, DIMENSION(1:10,1:20,ndim,zdim) :: a

! Promoted with ndim before and zdim after existing dimension
REAL, DIMENSION(ndim,1:10,1:20,zdim) :: a
! Promoted with ndim in middle and zdim after existing dimensions
REAL, DIMENSION(1:10,ndim,1:20,zdim) :: a
! Promoted with ndim before and zdim in middle existing dimensions
REAL, DIMENSION(1:10,1:20,ndim,zdim) :: a
\end{lstlisting}

\subsubsection{API}
Signature of the available method to perform a promotion is shown in 
Listing~\ref{lst:pri_promote}. The first parameter is a 
\lstinline|PromotionInfo| object describing the promotion to be performed. The
second parameter is the function definition in which the promotion is performed 
and finally the current translation unit.

\begin{lstlisting}[label=lst:pri_promote, language=java,
  title=Field.java - promote]
public static void promote(PromotionInfo fieldInfo, 
                           FfunctionDefinition fctDef,
                           XcodeProgram xcodeml)
\end{lstlisting}