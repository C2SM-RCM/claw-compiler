\chapter{Transformation}
\label{chapter:transformation}
A transformation class is the the basic representation of a manipulation of the
\gls{ast} triggered by a directive or on a whole translation unit. Each 
translation unit transformation or directive transformation is implemented in
its own class. If the directive can be used as a block directive (with a
\lstinline|!$end <directive>| directive), the class must inherits from the
\lstinline!ClawBlockTransformation!. Otherwise, it can inherits from the basic
\lstinline!ClawTransformation! class.

\section{Type of transformation}
\label{section:trans_type}
Transformations are divided into two distinct groups. The independent and the
dependent transformations. The first one, as its name implies, is performed
independently of any other transformations. Translation unit transformations 
are always independent. The dependent transformation, in the
other hand, is applied only when it can be combined with another dependent
transformation of the same kind in its group. Most of the transformations are
independent. The best example for the dependent transformation is the loop
fusion transformation. As shown in Listing \ref{lst:trans_dep}, there are two
CLAW directives (line 4 and 9). These directives will trigger two dependent loop
fusion transformations. Alone, those transformations have no effect
on the \gls{ast}. The translation engine will analyze if the
first one can be transformed with the second one. If so, the transformations will
be grouped together and the fusion will happen. Otherwise, the transformations
are just ignored as they have to depend on at least one other transformation. 
Checks that match dependent transformation together are specific to the kind
of transformation.

\lstinputlisting
  [
    label=lst:trans_dep,
    caption=Loop fusion example,
    language=Fortran
  ]{code/trans_dependent.f90}

\section{Trigger transformation}
\label{section:trans_trigger}
Transformations can be triggered by a directive or they can be applied for each
translation unit independently from the directives present in it or not.
This information is specified in the configuration file of the transformation
set (see \ref{section:conf_trans_set}).


\section{Transformation application order}
The first step of the translation of a translation unit (an XcodeML/F \gls{ir}),
is the detection of all the directives. Each directive will trigger the
creation of an instance of the transformation class it belongs to, if any. 
For example, a loop-fusion directive will trigger the creation of a 
\lstinline|LoopFusion| instance. On this instance, the \lstinline|analyze()| 
method is called in order to determine if the transformation can take place.
If the analysis step is successful, the transformation is added to its 
transformation group. Once all the transformation instances have been analyzed
and categorized by groups, the actual code transformation can take place on 
the XcodeML/F \gls{ir}. All transformations in a group are applied one after 
another in a FIFO order. The order in which groups are processed is determined 
by the CLAW configuration file.

\lstinputlisting
  [
    label=lst:config,
    caption=CLAW default configuration,
    language=xml
  ]{code/claw-default.xml}

As shown in Figure \ref{lst:config}, the transformation order is specified the
configuration file under the XML format.
Each transformation group name refers to its name in the transformation set it
belongs to. Global configuration and transformation set file are described in
details in Chapter \ref{chapter:configuration}.

\section{Add a new transformation}
\label{section:new_trans}
A transformation in the \clawfc is represented as a class that inherits from
the \lstinline|ClawTranformation| or \lstinline|ClawBlockTransformation|
base class. In order to add a new transformation into the the CLAW Compiler,
the following steps must be done.

\begin{enumerate}
\item Create a new class that inherits from one of the base transformation
      class.
\item (Optional) Define the directive that will trigger the transformation 
      and add it to the CLAW language parser if the transformation is 
      triggered by a directive.
\item (Optional) Detect and categorize the new transformation if the 
      transformation is triggered by a directive.
\item Add the transformation to a transformation set configuration and to
      the global configuration.
\end{enumerate}

As an example, the next 4 subsection describe those steps with more details.

\subsection{New transformation class}
The transformation created at this step will be a simple independent
transformation triggered by a directive. It will then inherits for 
\lstinline|ClawTransformation| base class.

\lstinputlisting
  [
    label=lst:myfirsttransformation.java,
    caption=MyFirstTransformation.java,
    language=java
  ]{code/MyFirstTransformation.java}

The transformation class shown in Listing \ref{lst:myfirsttransformation.java}
is really simple. It just inherits from \lstinline|ClawTransformation| on line
16. Therefore, it has to implements the \lstinline|analyze|,
\lstinline|transform| and \lstinline|canBeTransformedWith| methods as they are
abstract in the base class.

The \lstinline|analyze| method does not perform any check and just return
\lstinline|true| to tell the translation engine that the actual
transformation can occur.

The \lstinline|transform| method is pretty simple. It delete the pragma that
triggered the transformation from the \gls{ast}. It will then not be in the
resulting transformed code.

\subsection{Directive and CLAW language parser}
If the new transformation class is triggered by a directive, it needs to
be set up in the language parser in order to be instantiated. Listing
\ref{lst:mydirective} is the directive that is used in the current example.

\begin{lstlisting}[label=lst:mydirective, caption=Example directive, language=fortran]
!$claw mydirective
\end{lstlisting}

To parse the directive, two files need to be modified. First, the new directive
needs to be added to the \lstinline|ClawDirective| enumeration. It is added as
the last element of the enumeration in the Listing \ref{lst:clawdirective}.

\begin{lstlisting}[label=lst:clawdirective, caption=ClawDirective.java, language=java]
package cx2x.translator.language.base;

public enum ClawDirective {
  ARRAY_TRANSFORM,
  ARRAY_TO_CALL,
  DEFINE,
  IGNORE,
  KCACHE,
  LOOP_FUSION,
  LOOP_INTERCHANGE,
  LOOP_HOIST,
  LOOP_EXTRACT,
  PRIMITIVE,
  PARALLELIZE,
  REMOVE,
  VERBATIM,
  MYDIRECTIVE
}
\end{lstlisting}

Then, the CLAW language parser has to be modified to understand this new
directive and return the correct value from the enumeration.

\begin{lstlisting}[label=lst:clawdirectiveantlr, caption=Claw.g4, language=java]
grammar Claw;

@header
{
import cx2x.translator.common.ClawConstant;
import cx2x.translator.language.base.*;
import cx2x.translator.language.common.*;
import cx2x.translator.language.helper.target.Target;
import cx2x.translator.common.Utility;
}

/*----------------------------------------------------------------------------
 * PARSER RULES
 *----------------------------------------------------------------------------*/

// Entry point of the parsing
analyze returns [ClawLanguage l]
  @init{ $l = new ClawLanguage(); }:
  CLAW directive[$l] EOF
;

directive[ClawLanguage l]:
  // The new directive
  MYDIRECTIVE { $l.setDirective(ClawDirective.MYDIRETIVE); }
;

/*----------------------------------------------------------------------------
 * LEXER RULES
 *----------------------------------------------------------------------------*/

// Start point
CLAW         : 'claw';
MYDIRECTIVE  : 'mydirective';
\end{lstlisting}

In Listing \ref{lst:clawdirectiveantlr}, there is a minimal version of the
ANTLR grammar file of the CLAW language parser. The token
\lstinline|mydirective| is added in the lexer rules (line 33) and is then
used in the parser rules (line 24). ANTLR grammar file accepts Java code
to be inserted during the parsing. On line 24, the enumeration value for
the new directive is set in the analyzed pragma object.

\subsection{Detection and categorization}
The new directive has its transformation class and can be analyzed by the
CLAW language parser.

Now, it needs to be detected in order to create the transformation instance.
This happens in the \lstinline|ClawTranslator| class. The
\lstinline|generateTransformation| method detects all the pragmas in the 
translation unit and categorize them.

As shown in Listing \ref{lst:categorization}, a new case is added to the switch
statement that categorize the transformation. Here, a new instance of
\lstinline|MyFirstTransformation| transformation is created and added to the
correct group as long as its analysis step succeed.

\begin{lstlisting}[label=lst:categorization, caption=ClawXcodeMlTranslator.java, language=java]
switch(analyzedPragma.getDirective()) {
  case MYDIRECTIVE:
    addOrAbort(new MyFirstTransformation(analyzedPragma));
    break;
}
\end{lstlisting}


\subsection{Enable the new transformation}
Finally, the new transformation needs to be enabled by adding it to the
configuration file.

Listing \ref{lst:new_trans_set} shows the transformation set configuration
including the new transformation.
transformation class.


\begin{lstlisting}[label=lst:new_trans_set, caption=claw-dummy-set.xml, language=xml]
<transformations>
  <transformation name="delete-pragma" type="independent" trigger="directive"
    class="cx2x.translator.transformation.MyFirstTransformation"/>
</transformations>
\end{lstlisting}

Listing \ref{lst:new_trans_conf} uses the transformation set from Listing 
\ref{lst:new_trans_set} and enable its only transformation.

\begin{lstlisting}[label=lst:new_trans_conf, caption=claw-dummy.xml, language=xml]
<claw version="0.4">
  <global type="extension"/>
  <sets>
    <set name="claw-dummy"/>
  </sets>
  <groups>
    <group name="delete-pragma"/>
  </groups>
</claw>
\end{lstlisting}

\subsection{Test the new transformation}
The last step is to compile and install the new version of the \clawfcomp
containing the new directive and transformation class. Once it is done, it
can be directly tested with a simple FORTRAN program.

Listing \ref{lst:mydirectiveoriginal} is the original code. Once it is
transformed by the CLAW FORTRAN Compiler, it will produce the code
in Listing \ref{lst:mydirectivetransformed}. The command to call
the \clawfcomp is shown in Listing \ref{lst:clawfcexample}.

\lstinputlisting
  [
    label=lst:mydirectiveoriginal,
    caption=original.f90,
    language=fortran
  ]{code/mydirective_orig.f90}

\lstinputlisting
  [
    label=lst:mydirectivetransformed,
    caption=transformed.f90,
    language=fortran
  ]{code/mydirective_trans.f90}

\begin{lstlisting}[label=lst:clawfcexample, caption=Call the compiler, language=bash]
$ clawfc -o transformed.f90 original.f90
\end{lstlisting}
