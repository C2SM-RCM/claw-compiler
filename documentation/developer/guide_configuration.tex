\chapter{CLAW FORTRAN Compiler configuration}
\section{Transformation set configuration}
A transformation set is a group of transformations that are packed together and
that can be used during the translation. Each transformation set must be
described in a tranformation set configuration file.

\lstinputlisting
  [
    label=lst:internal_set_conf,
    caption=CLAW internal transformation set configuration,
    language=XML
  ]{code/claw-internal-set.xml}

A transformation set configuration defines several point of a transformation:
\begin{itemize}
  \item name: A unique name given to the transformation. This name is used when
        defining the application order of the transformations
  \item type: Type of the transformation as describe in section
        \ref{section:trans_type}. Only two values are possible
        \lstinline|dependent| or \lstinline|independent|.
  \item trigger: How the transformation is triggered as descrived in section
        \ref{section:trans_trigger}. Only two values are possible
        \lstinline|directive| or \lstinline|translation_unit|.
  \item class: Fully qualified class name implementing the transformation.
\end{itemize}

\section{Main configuration}
\subsection{Default configuration}
\subsection{User defined configuration}
