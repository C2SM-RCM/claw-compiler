\chapter{CLAW Compiler Configuration}
\label{chapter:configuration}

The different transformations enabled in the \clawfcomp and their order of
application is configurable in the \clawfcomp global configuration file. Each
transformation might also have different options that can be activated via this
configuration file. The available transformations are defined in transformation
set files (see Section~\ref{section:conf_trans_set}).

\section{Transformation Set}
\label{section:conf_trans_set}
A transformation set is a group of transformations that are packaged together 
and that can be used during the translation phase. Each transformation set must 
be described in a transformation set configuration file once.
Listing~\ref{lst:internal_set_conf} shows the transformation set
configuration file for the CLAW internal transformation set.

\lstinputlisting
  [
    label=lst:internal_set_conf,
    caption=CLAW internal transformation set configuration file,
    language=XML
  ]{code/claw-internal-set.xml}

A transformation set configuration file defines several characteristics of each
transformation defined into it:
\begin{itemize}
  \item \textbf{name}: A unique name given to the transformation. This name is
        used when defining the activation and the application order of the
        transformation.
  \item \textbf{type}: Type of the transformation as describe in
        Section~\ref{section:trans_type}. Only two values are possible
        \lstinline|dependent| or \lstinline|independent|.
  \item \textbf{trigger}: How the transformation is triggered as described in
        Section~\ref{section:trans_trigger}. Only two values are possible
        \lstinline|directive| or \lstinline|translation_unit|.
  \item \textbf{directive}: When \lstinline|directive| value is defined for the
        trigger option, this define which directive triggers it (e.g.
        \lstinline|acc|, \lstinline|omp| or \lstinline|claw|).
  \item \textbf{class}: Fully qualified class name implementing the
        transformation.
\end{itemize}

\subsection{External Transformation Set}
External transformations can be implemented separately from the CLAW Compiler
infrastructure and plugged into it easily. Section~\ref{section:new_trans}
explains in detail how to create and plug external transformations.

External transformations must be defined in an external transformation set
configuration file. This transformation set must specify the
name of the JAR file containing the transformation classes in the jar attribute
of the \lstinline|transformations| element as shown in
Figure~\ref{lst:external_set_conf}.

\lstinputlisting
  [
    label=lst:external_set_conf,
    caption=External transformation set configuration file,
    language=XML
  ]{code/external-set.xml}

The JAR files specified in external transformation sets can be located anywhere
on disk. These location have to be set in the \lstinline!CLAW_TRANS_SET_PATH!
environment variable.
Several paths can be provided and they must be separated by \lstinline!:!.


\section{\clawfcomp Global Configuration}
The global configuration gives information about which transformation sets are
used, which transformation are enabled, in which order the transformations are
applied and several configuration key-value pair that can be accessed by the
transformations them-self.

The configuration is separated in different sections:
\begin{itemize}
\item \textbf{global}: this section encapsulates several key-value pair
      configuration parameters. The attribute \lstinline!type! of the global
      element defines if the configuration is a root configuration or an
      extension of the default one. User can overwrite partially or entirely
      the default configuration.
      The only mandatory key-value pair in the global section is the
      \lstinline!translator! key.
\item \textbf{sets}: this section defines which transformation sets are used.
      The name corresponds to the filename of the transformation set
      configuration file without extension.
\item \textbf{groups}: this section defines which transformations are enabled.
      The name corresponds to the name defined in each transformation set
      configuration file.
      Only transformations part of the used transformation sets can be defined
      here.
      The order is defined by the order of declaration from top to bottom.
\end{itemize}


\subsection{Default Configuration}
\lstinputlisting
  [
    label=lst:claw_default_conf,
    caption=CLAW Compiler default configuration,
    language=XML
  ]{code/claw-default.xml}

\subsection{User Defined Configuration}
Listing~\ref{lst:external_config} is an example of a user configuration that
can be passed to the \clawfcomp. It uses one transformation set and enable
one transformation only. The transformation set is the one described in
Listing~\ref{lst:external_set_conf}. This configuration will overwrite
entirely the default configuration as its \lstinline!type! is \lstinline!root!.

\lstinputlisting
  [
    label=lst:external_config,
    caption=Example of user root configuration,
    language=XML
  ]{code/external_config.xml}

Listing~\ref{lst:extension_config} is an example of a user extension
configuration. Extension configuration overwrite only partially the default
configuration. In this example, only the \lstinline!groups! element is
overwritten with a new order for transformation.

\lstinputlisting
  [
    label=lst:extension_config,
    caption=Example of user extension configuration,
    language=XML
  ]{code/extension_config.xml}
