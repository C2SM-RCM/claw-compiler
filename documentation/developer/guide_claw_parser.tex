\chapter{CLAW directive language}
%TODO add ref for the document
The CLAW directive language is specified in the "CLAW directive language
specification". To validate and understand this language, a parser is needed
in the heart of the \clawfcomp.

\section{CLAW directive language parser}
%TODO add ref to ANTLR
The CLAW directive language parser is based on the ANTLR project
\cite{Parr:2013:DAR:2501720}. ANTLR is a parser generator. From a grammar file,
ANTLR generates a parser that is then used in the CLAW \xcodeml to \xcodeml
translator to interpret the directives.

\lstinputlisting
  [
    label=lst:antlr,
    language=Java,
    caption=ANTLR Grammar Example
  ]{code/clawp.g4}

Listing \ref{lst:antlr} is a minimalist ANTLR example that support only two
directives \lstinline|!$claw acc| or \lstinline|!$claw omp|.

In this grammar, there are two important sections, the parser and the lexer
sections. The lexer section defined what will be recognized as a token. Here
we defined the directives, the clauses but also more complex construct such as
the \lstinline|IDENTIFIER| on line 28. It described the possible element to
compose an identifier. This \lstinline|IDENTIFIER| can then be used later in
the parser rules.
The parser rules are defining the actual grammar of the language. In the case
of this small example, the grammar says that the directive string must begins
with \lstinline|claw| and then be followed by \lstinline|acc| or
\lstinline|omp|. The code in \lstinline|{}| is java code that is triggered
when the grammar rule is activated. In the example, it sets a value of the
object returned after the analysis.
The \lstinline|@init{}| section allows to execute some Java code before
triggering the actual parse. The \lstinline|@header{}| allows to insert Java
code before the parser class.

\begin{lstlisting}[label=lst:antlr_cmd, caption=ANTLR parser generation command, language=bash]
javac -classpath <antlr_jar> org.antlr.v4.Tool -o . -package cx2x.translator.language.parser Claw.g4
\end{lstlisting}

The full CLAW ANTLR grammar is defined in the following file:
\lstinline|omni-cx2x/src/cx2x/translator/language/parser/Claw.g4|
