\documentclass{article}
\usepackage{geometry}
\usepackage{xcolor}
\usepackage{listings}
\lstset{%
  backgroundcolor=\color{gray!40},
  breaklines=true,
  basicstyle=\ttfamily,
  commentstyle=\ttfamily,
  language=fortran,
  showspaces=false,
  showstringspaces=false
}
\usepackage{enumerate}

\title{CLAW Fortran Compiler Documentation}
\author{Valentin Clement}
\date{February 05, 2016\\\vspace{1em}v0.1}

\setlength\parindent{0pt}
\setlength{\parskip}{1em}

\begin{document}
\maketitle

CLAW Fortran Compiler is a source-to-source compiler interpreting the CLAW
directive language.

The CLAW Fortran Compiler is based on the OMNI Compiler. For more information 
about the OMNI Compiler project, visit http://omni-compiler.org.
 
OMNI Compiler user manual can be found here: http://omni-compiler.org/manual.html

For more information about the CLAW directive language, visit
https://github.com/C2SM-RCM/claw-language-specification


\section{Compiler options}
\begin{lstlisting}
usage: clawfc_test <OPTIONS> <INPUTFILE> ...

CLAW Fortran compiler options:

   -o <file>                  : place the output into <file>.
   -I <dir>                   : add the directory dir to the list of directories to be searched for header files.
   -J <dir>                   : specify where to put .mod and .xmod files for compiled modules.
   -D <dir>                   : specify output directory for transformed files
   -t=,--target=<target>      : specify the type of accelerator directive language for code generation.
   --target-list              : list the available type of accelerator directive language supported.
   -d=,--directive=<dir-lang> : specify the type of accelerator directive language for code generation.
   --directive-list           : list the available type of accelerator directive language supported.
   --config=<config_file>     : specify a different configuration for the translator.
   --show-config              : list the current configuration information. If used with --config, list the information from the specific configuration.
   -v,--verbose               : print processing status.
   --version                  : print version.
   -h,--help                  : print usage.
   --show-env                 : show environment variables.
   --no-dep                   : don't generate .mod or .xmod file for dependencies.
   -f,--force                 : force the translation of files without directives.
   --debug                    : save intermediate files in __omni_tmp__.
   --stop-pp                  : save intermediate files and stop after preprocess.
   --stop-dependencies        : save intermediate files and stop after dependencies resolution.
   --stop-frontend            : save intermediate files and stop after frontend.
   --stop-translator          : save intermediate files and stop after translator.

Decompiler options:
   -w <integer>               : Set the number of columns for the output Fortran file (default: 80).
   -l                         : Add preprocessor line directives in the output Fortran file.

Process Options

   --Wp[option] : Add preprocessor option.
   --Wf[option] : Add frontend option.
   --Wx[option] : Add Xcode translator option.
   --Wb[option] : Add backend option.
\end{lstlisting}

\subsection{Options for the output}
These options affect how the CLAW compiler writes the output files.

\textbf{\texttt{-o}}\\
The output file can be specified with the \texttt{-o} option. 
\begin{lstlisting}
$ clawfc -o transformed_code.f90 original_code.f90
\end{lstlisting}

\textbf{\texttt{-D}}\\
The option \texttt{-D} can be used to specify an output directory. If defined, all the file 
will be written in the specified directory

\begin{lstlisting}
$ clawfc -D src/ -o transformed_code.f90 original_code.f90
\end{lstlisting}

\subsection{Options for directory search}
These options affect how the CLAW compiler searches for files specified by the
\texttt{"INCLUDE"} directive and where it searches for previously compiled modules.

\textbf{\texttt{-I}}\\
These affect interpretation of the \texttt{"INCLUDE"} directive (as well as of the \texttt{"\#include"} directive of the cpp preprocessor).
All the specified paths are used to search for header files.
\begin{lstlisting}
$ clawfc -I mpi/ -o transformed_code.f90 original_code.f90
\end{lstlisting}

\textbf{\texttt{-J}}\\
All specified paths are used to search for \texttt{.mod/.xmod} file when previously compiled modules are required by a \texttt{"USE"} statement.
\begin{lstlisting}
$ clawfc -J xmod/ -o transformed_code.f90 original_code.f90
\end{lstlisting}

\subsection{Options to affect the workflow}
These options affect the workflow of the compiler. 

\textbf{\texttt{--no-dep}}\\
Do not generate module file (\texttt{.xmod}) for the dependencies. By default, module file for the dependencies are generated automatically. 
\begin{lstlisting}
$ clawfc --no-dep -o transformed_code.f90 original_code.f90
\end{lstlisting}

\textbf{\texttt{-f,--force}}\\
Force the compilation of files that do not have CLAW directives. By default, files without CLAW directives are ignored and just copied to the destination file. 
\begin{lstlisting}
$ clawfc --force -o transformed_code.f90 original_code.f90
\end{lstlisting}

\textbf{\texttt{--stop-pp}}\\
Stop the compiler after the preprocessing.
\begin{lstlisting}
$ clawfc --stop-pp -o transformed_code.f90 original_code.f90
\end{lstlisting}

\textbf{\texttt{--stop-dependencies}}\\
Stop the compiler after the dependencies resolution. Module files are generated but nothing is done with the input file. 
\begin{lstlisting}
$ clawfc --stop-dependencies -o transformed_code.f90 original_code.f90
\end{lstlisting}

\textbf{\texttt{--stop-frontend}}\\
Stop the compiler after the front-end. Input XcodeML file is generated. 
\begin{lstlisting}
$ clawfc --stop-frontend -o transformed_code.f90 original_code.f90
\end{lstlisting}

\textbf{\texttt{--stop-translator}}\\
Stop the compiler after the translation. Output XcodeML file is generated from the input XcodeML file with applied translation. 
\begin{lstlisting}
$ clawfc --stop-translator -o transformed_code.f90 original_code.f90
\end{lstlisting}

\subsection{Options for the accelerator language directive}
These options affect the generation of accelerator language directive.

\textbf{\texttt{-d=,--directive=<dir-lang>}}\\
Allow to specify the type of accelerator directive language for code generation.

\textbf{\texttt{--directive-list}}\\
Print the list of available type of accelerator directive language supported.

\subsection{Options for debugging}
These options affect the output at the command line and give information about the process. 

\textbf{\texttt{-v,--verbose}}\\
Print the inner commands run during the compilation process. 

\textbf{\texttt{--version}}\\
Print the version of the compiler. 

\textbf{\texttt{-h,--help}}\\
Print help about the compiler. 

\textbf{\texttt{--show-env}}\\
Show all the environment variables used in the compiler workflow. 

\textbf{\texttt{--debug}}\\
Print process information and store all the intermediate files in \_\_omni\_tmp\_\_ directory. 

\subsection{Options for subprocesses}
These options affect the different process part of the CLAW compiler. 

\textbf{\texttt{--Wp}}\\
Pass options to the underlying preprocessor. 
\begin{lstlisting}
$ clawfc --Wp-D_OPENACC -o transformed_code.f90 original_code.f90
\end{lstlisting}

\textbf{\texttt{--Wf}}\\
Pass options to the underlying front-end. 
\begin{lstlisting}
$ clawfc --Wf-no-module-cache -o transformed_code.f90 original_code.f90
\end{lstlisting}

\textbf{\texttt{--Wd}}\\
Pass options to the underlying translator. 
\begin{lstlisting}
$ clawfc --Wd-d -o transformed_code.f90 original_code.f90
\end{lstlisting}

\textbf{\texttt{--Wb}}\\
Pass options to the underlying preprocessor. 
\begin{lstlisting}
$ clawfc --Wb-w80 -o transformed_code.f90 original_code.f90
\end{lstlisting}

\end{document}
